% This file tests the emulation of gb4e using the gb4e-emulate package
\documentclass{article}
\usepackage{lipsum}
\usepackage{gb4e-emulate}

\setlist[exe,1]{leftmargin=*} %  Uncomment this line to have flush left examples
\begin{document}
\lipsum[1]
\begin{exe}
\ex\label{1}
\begin{xlist}
\ex[*]{ This is an example.}
\begin{xlist}
\ex[*]{This is a subexample}
\ex[]{This is a another subexample.}
\end{xlist}
\ex This is an example.
\ex{\gll Ceci est une example francaise.\\
		This is an example french\\
	\trans ``This is a french example.''}
\sn{An unnumbered example}
\end{xlist}
\end{exe}

\begin{exe}
\ex\label{2}
\begin{xlist}
\ex[*]{ This is an example.}
\begin{xlist}
\ex[*]{This is a subexample}
\ex[]{This is a another subexample.}
\end{xlist}
\ex This is an example.
\ex{\gll Ceci est une example francaise.\\
		This is an example french\\
	\trans ``This is a french example.''}
\end{xlist}
\end{exe}

\begin{exe}
\ex\label{3}
\begin{xlistI}
\ex{A ROMAN list}
\ex{Another one}
\end{xlistI}
\end{exe}

\begin{exe}
\ex\label{4}
\begin{xlistA}
\ex{An ALPHA list}
\ex{Another one}
\end{xlistA}
\end{exe}

\begin{exe}
\ex[]{Another}
\ex{Another}
\label{5}
\end{exe}

This is a \ref{5} to an example.
\lipsum[2]\footnote{As we can see in the following example:\begin{exe}[label=\roman*]\ex[*]{This is a fn example}\end{exe}}\lipsum[1]
\end{document}
